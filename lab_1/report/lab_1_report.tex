%% The comment character in TeX / LaTeX is the percent character.
%% The following chunk is called the header

\documentclass{article} % essential first line
\usepackage{times}    % this uses fonts which will look nice in PDF format
\usepackage{graphicx}   % needed for the figures
\usepackage{url}
\usepackage{adjustbox}
\usepackage{amsmath}
\usepackage{listings}
\usepackage{color}

\definecolor{dkgreen}{rgb}{0,0.6,0}
\definecolor{gray}{rgb}{0.5,0.5,0.5}
\definecolor{mauve}{rgb}{0.58,0,0.82}

\lstset{frame=tb,
  language=Java,
  aboveskip=3mm,
  belowskip=3mm,
  showstringspaces=false,
  columns=flexible,
  basicstyle={\small\ttfamily},
  numbers=none,
  numberstyle=\tiny\color{gray},
  keywordstyle=\color{blue},
  commentstyle=\color{dkgreen},
  stringstyle=\color{mauve},
  breaklines=true,
  breakatwhitespace=true,
  tabsize=3
}

%% Set the folder where pictures are located
\graphicspath{ {images/} }

%% Here I adjust the margins

\oddsidemargin -0.25in    % Left margin is 1in + this value
\textwidth 6.75in   % Right margin is not set explicitly
\topmargin 0in      % Top margin is 1in + this value
\textheight 9in     % Bottom margin is not set explicitly
\columnsep 0.25in   % separation between columns

%% Define a macro for inserting postscript images
%% ==============================================
%% This is a macro which nominally takes 3 parameters, 
%% it would be used as follows to insert and encapsulated postscript
%% image at the location where it is used.
%%
%% \EPSFIG{epsfilename}{caption}{label}
%% - epsfilename is the name of the encapsulated postscript file to be
%%               inserted at this location
%% - caption is the text to be shown as the figure caption, it will be
%%           prepended by Figure X.  The number X can be referenced
%%           using the label parameter.
%% - label is a name given to the figure, it can be referenced using the
%%         \ref{label} command.

%\def\EPSFIG[#1]#2#3#4{   % Don't be scared by this monsrosity
%\begin{figure}[hbt]    % it is a macro to save typing later
%\begin{center}     % 
%\includegraphics[#1]{#2} %
%\end{center}     %
%\caption{#3}     %
%\label{#4}     %
%\end{figure}     %
%}        %

%% Define the fields to be displayed by a \maketitle command
\author{Timothy Dee, Brent Barth}
%{\it Undergraduate, Department of Electrical and Computer Engineering, Iowa State University}
%\author{Brent Barth}
%\it Undergraduate, Department of Electrical and Computer Engineering, Iowa State University}
\title{Lab 1 Report}

%%
%% Header now finished
%%

\begin{document}    % Critical
\twocolumn
\thispagestyle{empty}   % Inhibit the page number on this page
\maketitle      % Use the \author, \title and \date info

%% Next comes the abstract, notice the curly-braces surrounding the
%% text.

\abstract{This report contains a description and analysis of the findings during Lab 1 of CprE 458, Real Time Systems. Contained herein is all relevant program code, and the results from having run this code.}

\section{Introduction}
This paper has sections discussing program code, energy consumption data analysis, and analysis of computation time. Each of these sections discuss design decisions and why those decisions were made.
TODO

\section{Part 1 - Code}
\subsection{High Performance Mode}
\begin{lstlisting}[float=*, caption={High Performance Mode},label={lst:HPM},numbers=left]
PMbutton2.setOnClickListener(new View.OnClickListener() {

      @Override
      public void onClick(View v) {

        OPERATIONmessage("[High Performance Mode] ###########################################");
        //TODO Please program for High Performance Mode here (done)

        DATAname = "1300000"; // Setting up the minimum frequency 1300 Mhz
        DATAaddress = "/sys/devices/system/cpu/cpu0/cpufreq/scaling_min_freq";
        ChangeCPUinfor(CPUname, DATAname, DATAaddress);
        DATAname = "1300000"; // Setting up the maximum frequency at 1300 MHz
        DATAaddress = "/sys/devices/system/cpu/cpu0/cpufreq/scaling_max_freq";
        ChangeCPUinfor(CPUname, DATAname, DATAaddress);

        CPUname = "High Power";
        DATAname = "current min frequency";
        DATAaddress = "/sys/devices/system/cpu/cpu0/cpufreq/scaling_min_freq";
        ReadCPUinfor(CPUname, DATAname, DATAaddress);
        DATAname = "current MAX frequency";
        DATAaddress = "/sys/devices/system/cpu/cpu0/cpufreq/scaling_max_freq";
        ReadCPUinfor(CPUname, DATAname, DATAaddress);

        OPERATIONmessage("[High Performance Mode] ###########################################");


      }
    });
\end{lstlisting}
Listing \ref{lst:HPM} describes our implementation of high performance mode. This code was straightforward due to its similarity with low performance mode.

\subsection{Dynamic Frequency Scaling Mode I}
\begin{lstlisting}[float=*,caption={Dynamic Frequency Scaling Mode I},label={lst:DFS_1},numbers=left]
private void setDFS_1(double cpu_load){
    // determine if I should go to a power mode
    if(cpu_load < .2){
      // go to low performance mode
      PMbutton1.performClick();
    }else if(cpu_load > .9){
      //go to high performance mode
      PMbutton2.performClick();
    }else{
      // set the frequency range between 51 Mhz and 1.3 Ghz
      DATAname = "51000"; // Setting up the minimum frequency 51 Mhz
      DATAaddress = "/sys/devices/system/cpu/cpu0/cpufreq/scaling_min_freq";
      ChangeCPUinfor(CPUname, DATAname, DATAaddress);
      DATAname = "1300000"; // Setting up the maximum frequency at 1300 MHz
      DATAaddress = "/sys/devices/system/cpu/cpu0/cpufreq/scaling_max_freq";
      ChangeCPUinfor(CPUname, DATAname, DATAaddress);

      CPUname = "Dynamic Mode";
      DATAname = "current min frequency";
      DATAaddress = "/sys/devices/system/cpu/cpu0/cpufreq/scaling_min_freq";
      ReadCPUinfor(CPUname, DATAname, DATAaddress);
      DATAname = "current MAX frequency";
      DATAaddress = "/sys/devices/system/cpu/cpu0/cpufreq/scaling_max_freq";
      ReadCPUinfor(CPUname, DATAname, DATAaddress);
    }
  }
\end{lstlisting}
Listing \ref{lst:DFS_1} describes our implementation of dynamic frequency scaling mode I. This mode varies the frequency range according to according to the CPU load. Lower processor frequencies will conserve power while higher processor frequencies will allow for increased performance. 
Certain CPU loads require that the processor be set to high performance mode or low performance mode. To accomplish this we use the performClick() method attached to the low performance mode and high performance mode buttons. This has the same effect as the user pressing the button which corresponds to these modes.

\subsection{Dynamic Frequency Scaling Mode II}
\begin{lstlisting}[float=*,caption={Dynamic Frequency Scaling Mode II},label={lst:DFS_2},numbers=left]
private void setDFS_2(){
    // get the battery level
    IntentFilter ifilter = new IntentFilter(Intent.ACTION_BATTERY_CHANGED);
    Intent batteryStatus = this.registerReceiver(null, ifilter);

    int level = batteryStatus.getIntExtra(BatteryManager.EXTRA_LEVEL, -1);
    int scale = batteryStatus.getIntExtra(BatteryManager.EXTRA_SCALE, -1);

    float battery_percent = level / (float)scale;

    // get the wireless radio state
    ConnectivityManager connectivityManager = (ConnectivityManager)
        this.getSystemService(Context.CONNECTIVITY_SERVICE);

    NetworkInfo network_info = connectivityManager.getActiveNetworkInfo();
    boolean is_wireless_on = (network_info != null);

    // get the charging state
    int status = batteryStatus.getIntExtra(BatteryManager.EXTRA_STATUS, -1);
    boolean is_charging = status == BatteryManager.BATTERY_STATUS_CHARGING ||
        status == BatteryManager.BATTERY_STATUS_FULL;

    //(continued on next page)    
\end{lstlisting}
\begin{lstlisting}[float=*,caption={Dynamic Frequency Scaling Mode II (continued)},label={lst:DFS_2_2},numbers=left]
// determine if we are charging
    if(is_charging){
      // if we are charging we don't care about power usage.... set to high performance mode
      PMbutton2.performClick();
    }else if(battery_percent < .3) {
      // we want to conserve energy because we are almost out of it, set to low performance mode
      PMbutton1.performClick();
    }else{
        // if we're not charging and not low battery we have some decisions to make
        // if the wireless radio is on we are likely doing something online.
        // If we are doing something we will like a more responsive device
        // so allow the processor to vary between two fairly high frequency states
        if (is_wireless_on) {
          // set to fairly high processing state
          // set the frequency range between 700 Mhz and 1.3 Ghz
          DATAname = "700000"; // Setting up the minimum frequency 51 Mhz
          DATAaddress = "/sys/devices/system/cpu/cpu0/cpufreq/scaling_min_freq";
          ChangeCPUinfor(CPUname, DATAname, DATAaddress);
          DATAname = "1300000"; // Setting up the maximum frequency at 1300 MHz
          DATAaddress = "/sys/devices/system/cpu/cpu0/cpufreq/scaling_max_freq";
          ChangeCPUinfor(CPUname, DATAname, DATAaddress);

          CPUname = "Dynamic Mode";
          DATAname = "current min frequency";
          DATAaddress = "/sys/devices/system/cpu/cpu0/cpufreq/scaling_min_freq";
          ReadCPUinfor(CPUname, DATAname, DATAaddress);
          DATAname = "current MAX frequency";
          DATAaddress = "/sys/devices/system/cpu/cpu0/cpufreq/scaling_max_freq";
          ReadCPUinfor(CPUname, DATAname, DATAaddress);
        } else {
          // set to lower processing state
          // set the frequency range between 51 Mhz and 700 Mhz
          DATAname = "51000"; // Setting up the minimum frequency 51 Mhz
          DATAaddress = "/sys/devices/system/cpu/cpu0/cpufreq/scaling_min_freq";
          ChangeCPUinfor(CPUname, DATAname, DATAaddress);
          DATAname = "700000"; // Setting up the maximum frequency at 1300 MHz
          DATAaddress = "/sys/devices/system/cpu/cpu0/cpufreq/scaling_max_freq";
          ChangeCPUinfor(CPUname, DATAname, DATAaddress);

          CPUname = "Dynamic Mode";
          DATAname = "current min frequency";
          DATAaddress = "/sys/devices/system/cpu/cpu0/cpufreq/scaling_min_freq";
          ReadCPUinfor(CPUname, DATAname, DATAaddress);
          DATAname = "current MAX frequency";
          DATAaddress = "/sys/devices/system/cpu/cpu0/cpufreq/scaling_max_freq";
          ReadCPUinfor(CPUname, DATAname, DATAaddress);
        }
    }
  }
\end{lstlisting}
Listings \ref{lst:DFS_2} and \ref{lst:DFS_2_2} show our implementation of dynamic frequency scaling mode 2. For this mode we though about the phone states which affect the scarcity of power. We determined that the wireless radio state and the charging state of the phone were significant enough to be determining factors in how we choose to manage the processor frequency.

In the case of the wireless radio the following train of logic was applied. If the wireless radio is on, then we are likely to be interacting with a web page. If we are interacting with a web page then we will certainly benefit from a more responsive device. Therefore we elect to put the phone in a higher power state while the wireless radio is on. We save energy by setting the phone in a lower power state when the wireless radio is off.
The decision making for the charging state is straightforward. If the phone is charging then we don't need to worry about power consumption, even if the battery is low.

\subsection{Dynamic Frequency Scaling Mode Mixed}
\begin{lstlisting}[float=*,caption={Dynamic Frequency Scaling Mode Mixed},label={lst:DFS_mixed},numbers=left]
  private void setDFS_Mixed(double cpu_load) {
    // get the battery level
    IntentFilter ifilter = new IntentFilter(Intent.ACTION_BATTERY_CHANGED);
    Intent batteryStatus = this.registerReceiver(null, ifilter);

    int level = batteryStatus.getIntExtra(BatteryManager.EXTRA_LEVEL, -1);
    int scale = batteryStatus.getIntExtra(BatteryManager.EXTRA_SCALE, -1);

    float battery_percent = level / (float)scale;

    // get the charging state
    int status = batteryStatus.getIntExtra(BatteryManager.EXTRA_STATUS, -1);
    boolean is_charging = status == BatteryManager.BATTERY_STATUS_CHARGING ||
        status == BatteryManager.BATTERY_STATUS_FULL;

    // determine if we are charging
    if(is_charging){
      // if we are charging we don't care about power usage.... set to high performance mode
      PMbutton2.performClick();
    }else {
      // vary the processor frequency based on the battery level.
      // the higher the battery level, the higher the frequency.
      // anywhere from 100000 khz to 1300000 khz
      double step_size = 1.0/12.0;

      // pretend like the cpu_load is less if the battery percent is lower
      int load_fraction = new Double(Math.ceil( (cpu_load * battery_percent) / step_size)).intValue();

      // load_fraction is at greatest 1
      int high_frequency = 100000 * (Integer.valueOf(load_fraction) + 1);
      int low_frequency = high_frequency;

      // set to fairly high processing state
      DATAname = String.valueOf(low_frequency); // Setting up the minimum frequency at low frequency
      DATAaddress = "/sys/devices/system/cpu/cpu0/cpufreq/scaling_min_freq";
      ChangeCPUinfor(CPUname, DATAname, DATAaddress);
      DATAname = String.valueOf(high_frequency); // Setting up the maximum frequency at high frequency
      DATAaddress = "/sys/devices/system/cpu/cpu0/cpufreq/scaling_max_freq";
      ChangeCPUinfor(CPUname, DATAname, DATAaddress);

      CPUname = "Dynamic Mode";
      DATAname = "current min frequency";
      DATAaddress = "/sys/devices/system/cpu/cpu0/cpufreq/scaling_min_freq";
      ReadCPUinfor(CPUname, DATAname, DATAaddress);
      DATAname = "current MAX frequency";
      DATAaddress = "/sys/devices/system/cpu/cpu0/cpufreq/scaling_max_freq";
      ReadCPUinfor(CPUname, DATAname, DATAaddress);
    }
  }
\end{lstlisting}
Listing \ref{lst:DFS_mixed} shows our implementation of dynamic frequency scaling mode mixed. For this mode we choose to use the charging state, battery charge percent, and cpu load as determining factors in our algorithm. All of these factors have a significant impact on how much we care about our energy usage. In many cases we are willing to sacrifice energy in exchange for performance.

As discussed previously, the charging state impacts our decisions in a straightforward way. If the device is being charged than we do not have to worry about power consumption, and we can activate high performance mode. When the device is not charging set a frequency range as a function of charge percent and cpu load.
We set load fraction to be an integer ranging between 0 and 12 depending on the cpu load. A load fracting of 0.0 yields 100000 Khz processor frequency while a load fraction of 12 yields 1300000 Khz processor frequency. load fraction increases linearly with cpu load and decreases linearly with battery percent.
battery percent causes the program to pretend like the cpu load is smaller than reality.

\subsection{Measuring Computation Time}
\begin{lstlisting}[float=*,caption={Measuring Computation Time},label={lst:MCT},numbers=left]
  //TODO
\end{lstlisting}
Listing \ref{lst:MCT} depicts the code we used to measure computation time under each of the different modes.
TODO

%we should say how we did each of the tests. We should also say what we used to generate each of the loads.
\section{Part 2 - Data Analysis}
%possibly insert a table indicating what energy consumption we got for each energy scheme under each load.
%might also be useful to include description of the testing.
%must include graphs of each test result
\subsection{Measuring energy consumption}
TODO

\subsection{Comparison and Analysis}
TODO

\section{Extra Part - Measuring Computation Time}
TODO

% we should sum up the results
\section{Conclusion}
TODO

% example figure
\begin{figure}[!hbt]
\begin{center}
\includegraphics[width=.4\textwidth,keepaspectratio]{example.png}
\end{center}
\caption{IR-Transmitter}
\label{FIG-TRANSMITTER}
\end{figure}

% example equation
\begin{equation}
\begin{split}
\frac{V_{out}-V_-}{R2}-\frac{V_--V_{in}}{R1}=0 \\
R1(V_{out}-V_-)=R2(V_--V_{in}) \\
V_{out}=\frac{R1V_-+R2V_--R2V{in}}{R1} \\
V_-=0 \\
V_{out}=\frac{-V_{in}R2}{R1} \\
\frac{V_{out}}{V_{in}}=\frac{-R2}{R1} \\ \\
\hline \\
\frac{V_{out}}{V_{in}}=\frac{47000}{47}=1000 \\
\end{split}
\label{eq-example}
\end{equation}
Equation \ref{eq-example} demonstrates example equation.

%% This bit generates the references.  This part starts to get
%% slightly tricky.
\bibliographystyle{unsrt} % Order by citation
\bibliography{report}

\end{document}