%% The comment character in TeX / LaTeX is the percent character.
%% The following chunk is called the header

\documentclass{article}	% essential first line
\usepackage{times}		% this uses fonts which will look nice in PDF format
\usepackage{graphicx}		% needed for the figures
\usepackage{url}
\usepackage{adjustbox}
\usepackage{amsmath}

%% Set the folder where pictures are located
\graphicspath{ {images/} }

%% Here I adjust the margins

\oddsidemargin -0.25in		% Left margin is 1in + this value
\textwidth 6.75in		% Right margin is not set explicitly
\topmargin 0in			% Top margin is 1in + this value
\textheight 9in			% Bottom margin is not set explicitly
\columnsep 0.25in		% separation between columns

%% Define a macro for inserting postscript images
%% ==============================================
%% This is a macro which nominally takes 3 parameters, 
%% it would be used as follows to insert and encapsulated postscript
%% image at the location where it is used.
%%
%% \EPSFIG{epsfilename}{caption}{label}
%% - epsfilename is the name of the encapsulated postscript file to be
%%               inserted at this location
%% - caption is the text to be shown as the figure caption, it will be
%%           prepended by Figure X.  The number X can be referenced
%%           using the label parameter.
%% - label is a name given to the figure, it can be referenced using the
%%         \ref{label} command.

%\def\EPSFIG[#1]#2#3#4{		% Don't be scared by this monsrosity
%\begin{figure}[hbt]		% it is a macro to save typing later
%\begin{center}			% 
%\includegraphics[#1]{#2}	%
%\end{center}			%
%\caption{#3}			%
%\label{#4}			%
%\end{figure}			%
%}				%

%% Define the fields to be displayed by a \maketitle command
\author{Timothy Dee, Brent Barth}
%{\it Undergraduate, Department of Electrical and Computer Engineering, Iowa State University}
%\author{Brent Barth}
%\it Undergraduate, Department of Electrical and Computer Engineering, Iowa State University}
\title{Lab 1 Report}

%%
%% Header now finished
%%

\begin{document}		% Critical
\twocolumn
\thispagestyle{empty}		% Inhibit the page number on this page
\maketitle			% Use the \author, \title and \date info

%% Next comes the abstract, notice the curly-braces surrounding the
%% text.

\abstract{This report contains a description and analysis of the findings durring Lab 1 of CprE 458, Real Time Systems.}

\section{Introduction}
TODO

\section{Part 1 - Code}
\subsection{High Performance Mode}
TODO
\subsection{Dynamic Frequency Scaling Mode I}
TODO
\subsection{Dynamic Frequency Scaling Mode II}
TODO
\subsection{Dynamic Frequency Scaling Mode Mixed}
TODO

%we should say how we did each of the tests. We should also say what we used to generate each of the loads.
\section{Part 2 - Data Analysis}
%possibly insert a table indicating what energy consumption we got for each energy scheme under each load.
%might also be useful to include description of the testing.
%must include graphs of each test result
\subsection{Measuring energy consumption}
TODO

\subsection{Comparison and Analysis}
TODO

\section{Extra Part - Measuring Computation Time}
TODO

% example figure
\begin{figure}[!hbt]
\begin{center}
\includegraphics[width=.4\textwidth,keepaspectratio]{example.png}
\end{center}
\caption{IR-Transmitter}
\label{FIG-TRANSMITTER}
\end{figure}

% example equation
\begin{equation}
\begin{split}
\frac{V_{out}-V_-}{R2}-\frac{V_--V_{in}}{R1}=0 \\
R1(V_{out}-V_-)=R2(V_--V_{in}) \\
V_{out}=\frac{R1V_-+R2V_--R2V{in}}{R1} \\
V_-=0 \\
V_{out}=\frac{-V_{in}R2}{R1} \\
\frac{V_{out}}{V_{in}}=\frac{-R2}{R1} \\ \\
\hline \\
\frac{V_{out}}{V_{in}}=\frac{47000}{47}=1000 \\
\end{split}
\label{eq-example}
\end{equation}
Equation \ref{eq-example} demonstrates example equation.

%% This bit generates the references.  This part starts to get
%% slightly tricky.
\bibliographystyle{unsrt}	% Order by citation
\bibliography{report}

\end{document}
